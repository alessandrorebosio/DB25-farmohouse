\documentclass[a4paper,12pt]{report}

% Codifica, lingua, font
\usepackage[utf8]{inputenc}
\usepackage[T1]{fontenc}
\usepackage[italian]{babel}
\usepackage{lmodern}

% Impaginazione
\usepackage{geometry}

% Grafica, colori, tabelle, float
\usepackage{graphicx}
\usepackage{float}
\usepackage[table]{xcolor}
\usepackage{tabularx}

% Matematica
\usepackage{amsmath}

% Verbatim e listing
\usepackage{fancyvrb}
\usepackage{alltt}
\usepackage{listings}

% Liste
% \usepackage{enumitem}

% Link e riferimenti intelligenti (hyperref prima di cleveref)
\usepackage{hyperref}
% \usepackage[italian]{cleveref}

\geometry{margin=1in}

% colori
\definecolor{codegray}{rgb}{0.5,0.5,0.5}
\definecolor{codeBlue}{HTML}{6495ED}
\definecolor{codegreen}{HTML}{12911F}
\definecolor{backcolour}{rgb}{0.95,0.95,0.92}

% stile listings
\lstdefinestyle{sqlstyle}{
    language=SQL,
    backgroundcolor=\color{backcolour},
    commentstyle=\color{codegreen}\itshape,
    keywordstyle=\color{codeBlue}\bfseries,
    numberstyle=\tiny\color{codegray},
    stringstyle=\color{codeBlue},
    basicstyle=\footnotesize\ttfamily,
    breakatwhitespace=false,
    breaklines=true,
    captionpos=b,
    keepspaces=true,
    numbers=left,
    numbersep=10pt,
    showspaces=false,
    showstringspaces=false,
    showtabs=false
}

% ambiente dedicato
\lstnewenvironment{sqlcode}[1][]{
    \lstset{style=sqlstyle,#1}
}{}

\hypersetup{
    colorlinks=false,
    pdfborder={1 1 1},
    linkbordercolor={1 0 0},
    urlbordercolor={1 0 0},
    citebordercolor={1 0 0},
    pdftitle={Elaborato Basi di Dati},
    pdfauthor={Maisam Noumi, Alessandro Rebosio, Filippo Ricciotti}
}

\title{
    \vspace*{2cm}
    \LARGE Relazione per il corso di \\[0.5cm]
    \textit{Basi di Dati} \\[2cm]
    \Huge\textbf{Agriturismo} \\[2cm]
}

\author{
    \Large
    Alessandro Rebosio \\
    Filippo Ricciotti
}

\date{
    \vspace{1cm}
    \today \\[0.5cm]
    Anno Accademico 2024-2025
}

\begin{document}

\maketitle

\tableofcontents

\chapter{Analisi dei requisiti}
\section{Intervista}

L'agriturismo intende dotarsi di una piattaforma digitale che razionalizzi le attività quotidiane 
e migliori l'esperienza dei clienti, integrando in un unico ambiente la gestione del personale, 
la vendita di prodotti e la promozione di eventi. Il titolare desidera uno strumento accessibile 
via web, utilizzabile da utenti registrati e dal personale autorizzato, in grado di offrire una 
visione chiara e centralizzata delle informazioni operative, riducendo errori e tempi di 
coordinamento.

Il cuore dell'applicativo è costituito da un catalogo di prodotti e da un calendario di
eventi, visibili ai visitatori e consultabili dagli utenti registrati. I prodotti, identificati 
da un codice univoco, sono descritti da un nome e da un prezzo, con la garanzia che i valori 
economici rimangano sempre positivi. Gli eventi, invece, sono presentati con titolo, 
descrizione, data di svolgimento e un numero di posti disponibili; la loro pubblicazione è 
effettuata da dipendenti autorizzati, così da mantenere controllo e coerenza dell'offerta.

Gli utenti potranno creare un account fornendo un nome utente, un indirizzo email e una 
password; ogni profilo sarà associato a una persona identificata tramite codice fiscale, 
così da assicurare un'anagrafica pulita e non ridondante. Una volta autenticati, gli utenti 
potranno consultare il catalogo, comporre ordini di acquisto di prodotti e completarne la 
registrazione: ogni ordine sarà tracciato con data e ora, e conterrà le righe di dettaglio con 
quantità e prezzo unitario, in modo da consentire il calcolo del totale e la successiva 
rendicontazione. Gli acquisti rimarranno associati in modo permanente all'account dell'utente, 
così da poterli rivedere e analizzare nel tempo.

Per la dimensione esperienziale dell'agriturismo, la piattaforma offrirà una sezione dedicata 
agli eventi: gli utenti interessati potranno iscriversi indicando il numero di partecipanti; il 
sistema dovrà garantire che le prenotazioni non superino i posti disponibili e registrerà 
automaticamente data e ora di ciascuna iscrizione. In questo modo, il titolare potrà monitorare 
in tempo reale l'andamento delle adesioni e prevedere l'affluenza, ottimizzando l'organizzazione 
delle serate e delle attività tematiche.

La gestione del personale rappresenta un altro pilastro del sistema. Ciascun dipendente sarà un 
utente abilitato a funzioni specifiche e caratterizzato da un ruolo (ad esempio sala, cucina, 
reception), con la possibilità di tracciarne lo storico delle variazioni nel tempo. La 
pianificazione dei turni avverrà attraverso la definizione di modelli di turno (per giorno della 
settimana, con orari di inizio e fine) e la loro assegnazione a calendario per una data 
specifica. Ogni assegnazione prevede uno stato — programmato, completato o assente — così da 
fotografare l'effettiva presenza; inoltre, il sistema eviterà conflitti, impedendo che uno 
stesso dipendente risulti assegnato a più turni nella medesima giornata.

Dal punto di vista direzionale, il titolare richiede una reportistica essenziale ma 
affidabile: l'andamento delle vendite per periodo, la partecipazione agli eventi e un quadro 
della presenza/assenza del personale sui turni. La piattaforma dovrà salvaguardare la sicurezza 
dei dati, conservando le password in forma sicura e applicando vincoli di integrità su prezzi e 
quantità; le operazioni frequenti — come consultare il catalogo, registrare un ordine o 
iscriversi a un evento — dovranno risultare rapide e semplici, privilegiando chiarezza e 
immediatezza d'uso.

\section{Estrazione dei concetti principali}

L'agriturismo intende realizzare una piattaforma digitale che unisca in un unico ecosistema la 
vendita di prodotti, la promozione e gestione degli eventi e l'organizzazione del personale. Il 
sistema sarà accessibile via web agli utenti registrati e al personale autorizzato, con 
l'obiettivo di offrire una vista centralizzata e coerente delle attività quotidiane, riducendo 
errori operativi e tempi di coordinamento. Il cuore dell'applicazione è rappresentato da un 
\textbf{catalogo di \underline{prodotti}} e da un calendario \underline{eventi}: i \underline{prodotti}, identificati in modo univoco (\textbf{codice}), e 
descritti da \textbf{nome} e \textbf{prezzo}, saranno acquistabili dagli \underline{utenti} autenticati; gli \underline{eventi}, 
caratterizzati da \textbf{titolo}, \textbf{descrizione}, \textbf{data} e \textbf{posti disponibili}, saranno visibili e prenotabili 
secondo regole di capienza stabilite dall'azienda.

Gli \underline{utenti} potranno creare un account fornendo \textbf{nome utente}, \textbf{email} e \textbf{password}; ogni account sarà 
associato a una \underline{persona} identificata da \textbf{codice fiscale}, in modo da mantenere un'anagrafica solida 
e priva di duplicati. Una volta autenticati, gli \underline{utenti} potranno consultare il catalogo e 
comporre \underline{ordini}, che verranno registrati con \textbf{data} e \textbf{ora} e articolati in \underline{righe d'ordine} di dettaglio con 
\textbf{quantità} e \textbf{prezzo unitario}, garantendo la correttezza dei totali e la tracciabilità nel 
tempo. Gli \underline{acquisti} resteranno permanentemente associati al profilo dell'\underline{utente}, consentendo 
storicizzazione e successive analisi gestionali.

La dimensione esperienziale sarà supportata da un modulo \underline{eventi}: la creazione degli \underline{eventi} è 
affidata a \underline{dipendenti} autorizzati e prevede l'indicazione dei \textbf{posti disponibili}. Gli \underline{utenti} 
potranno iscriversi (\underline{iscrizione evento}) specificando il \textbf{numero di partecipanti}, mentre il sistema dovrà prevenire 
il superamento della capienza e registrare automaticamente \textbf{data} e \textbf{ora} di ogni iscrizione. In 
parallelo, la gestione interna del \underline{personale} è fondata su \underline{ruoli} e \underline{turni}: ogni \underline{dipendente} possiede 
un \textbf{ruolo} corrente, con storico delle variazioni per fini di audit, e partecipa a una 
pianificazione che combina \underline{modelli di turno} (\textbf{giorno della settimana}, \textbf{nome}, \textbf{orari}) con 
\underline{assegnazioni di turno} a calendario per \textbf{date} specifiche. Ogni assegnazione registra lo \textbf{stato} 
effettivo (\textbf{programmato}, \textbf{completato}, \textbf{assente}) e impedisce conflitti, evitando che un \underline{dipendente} 
risulti pianificato su più turni nello stesso giorno.

A livello trasversale, la piattaforma tutela integrità e sicurezza dei dati: \textbf{prezzi} e \textbf{quantità} 
devono essere sempre positivi, le relazioni fra \underline{utenti}, \underline{dipendenti}, \underline{ordini} ed \underline{eventi} rispettano 
vincoli referenziali, e le informazioni sensibili come le \textbf{password} sono gestite in modo 
sicuro. Il \underline{titolare} dispone di una visione complessiva tramite report essenziali su \underline{vendite}, 
\underline{adesioni agli eventi} e \underline{presenza del personale}, mentre l'interfaccia privilegia semplicità e 
rapidità nelle operazioni più frequenti.

\appendix
\chapter{Guida Utente}

\section{Clonazione del repository}
Clonare il progetto da GitHub e accedere alla cartella:

\begin{verbatim}
> git clone https://github.com/alessandrorebosio/D25-farmhouse.git
> cd DB25-farmhouse
\end{verbatim}

\section{Installazione delle dipendenze}

Si consiglia di utilizzare un ambiente virtuale Python per isolare le dipendenze del progetto.

\begin{verbatim}
> python3 -m venv venv
\end{verbatim}

\noindent Attivazione dell'ambiente virtuale
\begin{verbatim}
# Su Linux/macOS:
> source venv/bin/activate
# Su Windows:
> venv\Scripts\activate
\end{verbatim}

\noindent Installazione delle dipendenze dal file requirements.txt
\begin{verbatim}
> pip install -r requirements.txt
\end{verbatim}

\section{Creazione del database}

Per creare il database MySQL a partire dagli script SQL forniti, assicurarsi di avere MySQL
installato e in esecuzione.

\begin{verbatim}
> mysql -u root -p < app/sql/db.sql
> mysql -u root -p < app/sql/demo.sql
\end{verbatim}

Verrà richiesta la password dell'utente \texttt{root}. Il comando eseguirà tutte le
istruzioni SQL contenute nel file \texttt{db.sql}, creando tabelle, vincoli e dati di
esempio necessari per l'applicazione.

\section{Avvio dell'applicazione}

Per avviare l'applicazione Django, assicurarsi che l'ambiente virtuale sia attivo e che il database
sia stato creato correttamente.

\begin{verbatim}
> python manage.py migrate
> python manage.py runserver
\end{verbatim}

L'applicazione sarà accessibile all'indirizzo \url{http://localhost:8000/} tramite browser. Effettuare
il login o la registrazione per iniziare a utilizzare il sistema.

\end{document}
