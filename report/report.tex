\documentclass[a4paper,12pt]{report}

% Codifica, lingua, font
\usepackage[utf8]{inputenc}
\usepackage[T1]{fontenc}
\usepackage[italian]{babel}
\usepackage{lmodern}

% Impaginazione
\usepackage{geometry}

% Grafica, colori, tabelle, float
\usepackage{graphicx}
\usepackage{float}
\usepackage[table]{xcolor}
\usepackage{tabularx}

% Matematica
\usepackage{amsmath}

% Verbatim e listing
\usepackage{fancyvrb}
\usepackage{alltt}
\usepackage{listings}

% Liste
% \usepackage{enumitem}

% Link e riferimenti intelligenti (hyperref prima di cleveref)
\usepackage{hyperref}
% \usepackage[italian]{cleveref}

\geometry{margin=1in}

% colori
\definecolor{codegray}{rgb}{0.5,0.5,0.5}
\definecolor{codeBlue}{HTML}{6495ED}
\definecolor{codegreen}{HTML}{12911F}
\definecolor{backcolour}{rgb}{0.95,0.95,0.92}

% stile listings
\lstdefinestyle{sqlstyle}{
  language=SQL,
  backgroundcolor=\color{backcolour},
  commentstyle=\color{codegreen}\itshape,
  keywordstyle=\color{codeBlue}\bfseries,
  numberstyle=\tiny\color{codegray},
  stringstyle=\color{codeBlue},
  basicstyle=\footnotesize\ttfamily,
  breakatwhitespace=false,
  breaklines=true,
  captionpos=b,
  keepspaces=true,
  numbers=left,
  numbersep=10pt,
  showspaces=false,
  showstringspaces=false,
  showtabs=false
}

% ambiente dedicato
\lstnewenvironment{sqlcode}[1][]{
  \lstset{style=sqlstyle,#1}
}{}

\hypersetup{
  colorlinks=false,
  pdfborder={1 1 1},
  linkbordercolor={1 0 0},
  urlbordercolor={1 0 0},
  citebordercolor={1 0 0},
  pdftitle={Elaborato Basi di Dati},
  pdfauthor={Maisam Noumi, Alessandro Rebosio, Filippo Ricciotti}
}

\title{
  \vspace*{2cm}
  \LARGE Relazione per il corso di \\[0.5cm]
  \textit{Basi di Dati} \\[2cm]
  \Huge\textbf{Agriturismo} \\[2cm]
}

\author{
  \Large
  Alessandro Rebosio \\
  Filippo Ricciotti
}

\date{
  \vspace{1cm}
  \today \\[0.5cm]
  Anno Accademico 2024-2025
}

\begin{document}

\maketitle

\tableofcontents

\chapter{Analisi dei requisiti}
\section{Intervista}

L'agriturismo intende dotarsi di una piattaforma digitale che
razionalizzi le attività quotidiane
e migliori l'esperienza dei clienti, integrando in un unico ambiente
la gestione del personale,
la vendita di prodotti e la promozione di eventi. Il titolare
desidera uno strumento accessibile
via web, utilizzabile da utenti registrati e dal personale
autorizzato, in grado di offrire una
visione chiara e centralizzata delle informazioni operative,
riducendo errori e tempi di
coordinamento.

Il cuore dell'applicativo è costituito da un catalogo di prodotti e
da un calendario di
eventi, visibili ai visitatori e consultabili dagli utenti
registrati. I prodotti, identificati
da un codice univoco, sono descritti da un nome e da un prezzo, con
la garanzia che i valori
economici rimangano sempre positivi. Gli eventi, invece, sono
presentati con titolo,
descrizione, data di svolgimento e un numero di posti disponibili; la
loro pubblicazione è
effettuata da dipendenti autorizzati, così da mantenere controllo e
coerenza dell'offerta.

Gli utenti potranno creare un account fornendo un nome utente, un
indirizzo email e una
password; ogni profilo sarà associato a una persona identificata
tramite codice fiscale,
così da assicurare un'anagrafica pulita e non ridondante. Una volta
autenticati, gli utenti
potranno consultare il catalogo, comporre ordini di acquisto di
prodotti e completarne la
registrazione: ogni ordine sarà tracciato con data e ora, e conterrà
le righe di dettaglio con
quantità e prezzo unitario, in modo da consentire il calcolo del
totale e la successiva
rendicontazione. Gli acquisti rimarranno associati in modo permanente
all'account dell'utente,
così da poterli rivedere e analizzare nel tempo.

Per la dimensione esperienziale dell'agriturismo, la piattaforma
offrirà una sezione dedicata
agli eventi: gli utenti interessati potranno iscriversi indicando il
numero di partecipanti; il
sistema dovrà garantire che le prenotazioni non superino i posti
disponibili e registrerà
automaticamente data e ora di ciascuna iscrizione. In questo modo, il
titolare potrà monitorare
in tempo reale l'andamento delle adesioni e prevedere l'affluenza,
ottimizzando l'organizzazione
delle serate e delle attività tematiche.

La gestione del personale rappresenta un altro pilastro del sistema.
Ciascun dipendente sarà un
utente abilitato a funzioni specifiche e caratterizzato da un ruolo
(ad esempio sala, cucina,
reception), con la possibilità di tracciarne lo storico delle
variazioni nel tempo. La
pianificazione dei turni avverrà attraverso la definizione di modelli
di turno (per giorno della
settimana, con orari di inizio e fine) e la loro assegnazione a
calendario per una data
specifica. Ogni assegnazione prevede uno stato — programmato,
completato o assente — così da
fotografare l'effettiva presenza; inoltre, il sistema eviterà
conflitti, impedendo che uno
stesso dipendente risulti assegnato a più turni nella medesima giornata.

Dal punto di vista direzionale, il titolare richiede una reportistica
essenziale ma
affidabile: l'andamento delle vendite per periodo, la partecipazione
agli eventi e un quadro
della presenza/assenza del personale sui turni. La piattaforma dovrà
salvaguardare la sicurezza
dei dati, conservando le password in forma sicura e applicando
vincoli di integrità su prezzi e
quantità; le operazioni frequenti — come consultare il catalogo,
registrare un ordine o
iscriversi a un evento — dovranno risultare rapide e semplici,
privilegiando chiarezza e
immediatezza d'uso.

\section{Estrazione dei concetti principali}

L'agriturismo intende realizzare una piattaforma digitale che unisca
in un unico ecosistema la
vendita di prodotti, la promozione e gestione degli eventi e
l'organizzazione del personale. Il
sistema sarà accessibile via web agli utenti registrati e al
personale autorizzato, con
l'obiettivo di offrire una vista centralizzata e coerente delle
attività quotidiane, riducendo
errori operativi e tempi di coordinamento. Il cuore dell'applicazione
è rappresentato da un
\textbf{catalogo di \underline{prodotti}} e da un calendario
\underline{eventi}: i
\underline{prodotti}, identificati in modo univoco (\textbf{codice}),
e descritti da
\textbf{nome} e \textbf{prezzo}, saranno acquistabili dagli
\underline{utenti} autenticati; gli
\underline{eventi}, caratterizzati da \textbf{titolo},
\textbf{descrizione}, \textbf{data} e
\textbf{posti disponibili}, saranno visibili e prenotabili secondo
regole di capienza stabilite
dall'azienda.

Gli \underline{utenti} potranno creare un account fornendo
\textbf{nome utente}, \textbf{email}
e \textbf{password}; ogni account sarà associato a una
\underline{persona} identificata da
\textbf{codice fiscale}, in modo da mantenere un'anagrafica solida e
priva di duplicati. Una
volta autenticati, gli \underline{utenti} potranno consultare il
catalogo e comporre
\underline{ordini}, che verranno registrati con \textbf{data} e
\textbf{ora} e articolati in
\underline{righe d'ordine} di dettaglio con \textbf{quantità} e
\textbf{prezzo unitario},
garantendo la correttezza dei totali e la tracciabilità nel tempo.
Gli \underline{acquisti}
resteranno permanentemente associati al profilo
dell'\underline{utente}, consentendo
storicizzazione e successive analisi gestionali.

La dimensione esperienziale sarà supportata da un modulo
\underline{eventi}: la creazione degli
\underline{eventi} è affidata a \underline{dipendenti} autorizzati e
prevede l'indicazione dei
\textbf{posti disponibili}. Gli \underline{utenti} potranno
iscriversi (\underline{iscrizione
evento}) specificando il \textbf{numero di partecipanti}, mentre il
sistema dovrà prevenire
il superamento della capienza e registrare automaticamente
\textbf{data} e \textbf{ora} di ogni
iscrizione. In parallelo, la gestione interna del
\underline{personale} è fondata su
\underline{ruoli} e \underline{turni}: ogni \underline{dipendente}
possiede un \textbf{ruolo}
corrente, con storico delle variazioni per fini di audit, e partecipa
a una pianificazione che
combina \underline{modelli di turno} (\textbf{giorno della
  settimana}, \textbf{nome},
\textbf{orari}) con \underline{assegnazioni di turno} a calendario
per \textbf{date} specifiche.
Ogni assegnazione registra lo \textbf{stato} effettivo (\textbf{programmato},
\textbf{completato}, \textbf{assente}) e impedisce conflitti, evitando che un
\underline{dipendente} risulti pianificato su più turni nello stesso giorno.

A livello trasversale, la piattaforma tutela integrità e sicurezza
dei dati: \textbf{prezzi} e
\textbf{quantità} devono essere sempre positivi, le relazioni fra
\underline{utenti},
\underline{dipendenti}, \underline{ordini} ed \underline{eventi}
rispettano vincoli
referenziali, e le informazioni sensibili come le \textbf{password}
sono gestite in modo sicuro.

Il \underline{titolare} dispone di una visione complessiva tramite
report essenziali su
\underline{vendite}, \underline{adesioni agli eventi} e
\underline{presenza del personale},
mentre l'interfaccia privilegia semplicità e rapidità nelle
operazioni più frequenti.

\chapter{Progettazione concettuale}
In questo capitolo presenteremo lo schema ER, partendo da una
versione iniziale e migliorandola
passo dopo passo ad arrivare a quella definitiva, attraverso dei raffinamenti.

\section{Schema iniziale}
Dopo aver eseguito l'analisi del dominio iniziale, abbiamo creato uno
schema di base con
le entità e le relazioni principali, che sarà poi perfezionato nei
passaggi successivi.
\begin{figure}[H]
  \centering
  \includegraphics[width=\textwidth, trim=200pt 170pt 50pt 0pt,
  clip]{./schemas/base.pdf}
  \caption{Schema ER iniziale}
  \label{fig:schema-iniziale}
\end{figure}

\section{Raffinamenti proposti}
\subsection{Utente e Dipendente}
Nel modello concettuale iniziale la \textbf{Persona} raggruppava
tutte le possibili interazioni
con il sistema: iscrizione, creazione di eventi, prenotazioni, ordini
e recensioni. Questo
approccio, sebbene corretto dal punto di vista logico, risultava poco
chiaro perché attribuiva
a un'unica entità responsabilità molto eterogenee.

\vspace{\baselineskip}
Per migliorare la rappresentazione è stato introdotto un raffinamento mediante
generalizzazione/specializzazione: la superclasse \textbf{Persona} è
stata mantenuta per
raccogliere gli attributi comuni (CF, nome, cognome), mentre le
funzionalità specifiche sono
state assegnate ai sottotipi \textbf{Cliente} e \textbf{Dipendente}.

\vspace{\baselineskip}
In questo modo i clienti gestiscono attività come acquisti,
recensioni, ordini e iscrizioni agli
eventi, mentre i dipendenti si occupano della creazione degli eventi
e della gestione dei
servizi. Tale raffinamento migliora la chiarezza semantica del
modello, riduce le ambiguità e
riflette meglio la separazione dei ruoli reali all'interno del
dominio applicativo.

\vspace{\baselineskip}
Il raffinamento mette in evidenza anche le dipendenze temporali (come
  la gestione dei turni \textbf{Shift} o la cronologia del personale
\textbf{Employee History}) e garantisce che ogni operazione rispetti
vincoli di consistenza e cardinalità, rendendo il modello complessivo
coerente, sicuro e facilmente estendibile.

\begin{figure}[H]
  \centering
  \includegraphics[width=\textwidth, trim=0 200pt 275pt 0,
  clip]{./schemas/refinements/user.pdf}
  \caption{Raffinamento utente e dipendente}
  \label{fig:raffinamento-utente}
\end{figure}

\newpage
\subsection{Prenotazione Servizi}
Nel modello iniziale i diversi tipi di servizi potevano essere
rappresentati come entità
distinte, con il rischio però di ridondanza e frammentazione dei dati.

\vspace{\baselineskip}
Con il raffinamento si è introdotta una \textbf{generalizzazione}: è
stata creata la superclasse
\textbf{Servizio}, che raccoglie gli attributi comuni (id, price,
type), mentre ciascuna
tipologia specifica di servizio (Camera e Ristorante) è modellata
come sottoclasse.

\vspace{\baselineskip}
Inoltre, è stato introdotto il legame con l'entità
\textbf{Prenotazione}, che consente di
registrare le informazioni su data di inizio e fine e di associare
ogni prenotazione a uno o
più servizi specifici tramite la relazione con \textbf{Dettagli
Prenotazione}. Questo
raffinamento permette di gestire correttamente scenari in cui un
utente prenota più servizi
differenti nello stesso arco temporale.
\begin{figure}[H]
  \centering
  \includegraphics[width=\textwidth, trim=0 200pt 300pt 0,
  clip]{./schemas/refinements/service.pdf}
  \caption{Raffinamento prenotazione e servizi}
  \label{fig:raffinamento-servizi-prenotazione}
\end{figure}

\newpage
\subsection{Prodotti e ordini}
Nel modello concettuale iniziale, la gestione degli ordini e dei
prodotti risultava poco
dettagliata: un ordine era semplicemente collegato a uno o più
prodotti, senza possibilità di
specificare informazioni aggiuntive come quantità o prezzo unitario.

\vspace{\baselineskip}
Con il raffinamento, è stata introdotta l'entità \textbf{Dettaglio
Ordine}, che funge da
associazione tra \textbf{Ordine} e \textbf{Prodotto}. Ogni dettaglio
ordine consente di
memorizzare, per ciascun prodotto incluso in un ordine, la quantità
acquistata e il prezzo
applicato. Questo permette di rappresentare in modo accurato scenari
reali come ordini
multiprodotto, applicazione di sconti o variazioni di prezzo nel tempo.

\vspace{\baselineskip}
Inoltre, viene mantenuta la generalizzazione tra \textbf{Persona} e
\textbf{Utente}, già
introdotta nei raffinamenti precedenti, per distinguere i dati
anagrafici comuni da quelli
specifici per l'accesso al sistema e la gestione degli ordini. Questo
approccio migliora la
flessibilità e la chiarezza del modello, consentendo una gestione più efficace
delle informazioni relative agli acquisti.
\begin{figure}[H]
  \centering
  \includegraphics[width=\textwidth, trim=0 300pt 325pt 0,
  clip]{./schemas/refinements/product.pdf}
  \caption{Raffinamento prodotti e ordini}
  \label{fig:raffinamento-prodotto-ordini}
\end{figure}

\newpage
\section{Schema concettuale finale}
Qui di seguito, è presente lo schema concettuale finale con tutti i
raffinamenti.

\begin{figure}[H]
  \centering
  \includegraphics[width=\textwidth, trim=0 0 0
  0]{./schemas/refinements/final.pdf}
  \caption{Schema ER, schema concettuale finale}
  \label{fig:schema-finale}
\end{figure}

\newpage
\chapter{Progettazione Logica}

\section{Stima del volume di dati}
Sulla base dei requisiti e dell'analisi di dominio, abbiamo stimato l'ordine
di grandezza dei dati iniziali e la crescita annua attesa per ciascuna
entità principale. La Tabella~\ref{tab:volume-dati} riassume i volumi e la
crescita previsti; i valori sono indicativi e utili per dimensionare indici,
partizionamento e politiche di archiviazione.

\begin{table}[H]
  \centering
  \begin{tabularx}{\textwidth}{|X|c|c|}
    \hline
    \rowcolor{gray!15}
    \textbf{Tabella} & \textbf{Volume stimato} & \textbf{E/A} \\
    \hline
    \texttt{ORDER} & 15\,000 & E \\
    \texttt{RESERVATION} & 8\,000 & E \\
    \texttt{PERSON} & 6\,000 & E \\
    \texttt{USER} & 5\,000 & E \\
    \texttt{REVIEW} & 3\,000 & E \\
    \texttt{PRODUCT} & 500 & E \\
    \texttt{EVENT} & 150 & E \\
    \texttt{SERVICE} & 100 & E \\
    \texttt{EMPLOYEE} & 50 & E \\
    \texttt{ROOM} & 30 & E \\
    \texttt{SHIFT} & 20 & E \\
    \texttt{RESTAURANT} & 5 & E \\
    \texttt{ORDER\_DETAIL} & 45\,000  & A \\
    \texttt{EMPLOYEE\_SHIFT} & 18\,000 & A \\
    \texttt{RESERVATION\_DETAIL} & 12\,000 & A \\
    \texttt{SUBSCRIPTION} & 2\,500 & A \\
    \texttt{EMPLOYEE\_HISTORY} & 80 & A \\
    \hline
  \end{tabularx}

  \caption{Stima volumi e classificazione Entità (E) o Associazione (A)}
  \label{tab:volume-dati}
\end{table}

\section{Descrizione operazioni}
\begin{table}[H]
  \centering
  \small
  \renewcommand{\arraystretch}{1.12}
  \begin{tabularx}{\textwidth}{|c|>{\raggedright\arraybackslash}X|c|c|}
    \hline
    \rowcolor{gray!20}
    \textbf{\#} & \textbf{Operazione} & \textbf{Op / 7gg} & \textbf{Tipo Utente} \\
    \hline
    1 & \hyperref[op1]{Registrazione nuovo utente} & 50 & Cliente \\
    \hline
    2 & \hyperref[op2]{Autenticazione / login} & 350 & Tutti (Clienti, Staff) \\
    \hline
    3 & \hyperref[op3]{Prenotazione servizio} & 125 & Cliente / Reception \\
    \hline
    4 & \hyperref[op4]{Modifica / cancellazione prenotazione} & 30 & Cliente / Reception \\
    \hline
    5 & \hyperref[op5]{Gestione/aggiunta prodotti} & 3 & Staff / Admin \\
    \hline
    6 & \hyperref[op6]{Visualizzazione storico ordini} & 15 & Cliente \\
    \hline
    7 & \hyperref[op7]{Inserimento recensione} & 40 & Cliente \\
    \hline
    8 & \hyperref[op8]{Gestione inventario / prodotti} & 10 & Staff \\
    \hline
    9 & \hyperref[op9]{Creazione evento} & 1 & Staff / Admin \\
    \hline
    10 & \hyperref[op10]{Iscrizione evento} & 15 & Cliente \\
    \hline
    11 & \hyperref[op11]{Creazione ordine} & 25 & Cliente / Staff \\
    \hline
    12 & \hyperref[op12]{Assegnazione / modifica turno dipendenti} & 1 & Admin \\
    \hline
    13 & \hyperref[op15]{Controllo disponibilità servizi (ricerca/filtri)} & 550 & Tutti \\
    \hline
    14 & \hyperref[op16]{Generazione report finanziari} & 1 & Admin \\
    \hline
    15 & \hyperref[op17]{Gestione sottoscrizioni eventi} & 5 & Staff / Admin \\
    \hline
    16 & \hyperref[op18]{Pianificazione turni dipendenti} & 3 & Admin \\
    \hline
    17 & \hyperref[op19]{Calcolo tasso occupazione camere} & 1 & Admin \\
    \hline
    18 & \hyperref[op20]{Analisi eventi più popolari} & 1 & Admin / Staff \\
    \hline
    19 & \hyperref[op21]{Individuazione prodotti più venduti} & 1 & Admin \\
    \hline
    20 & \hyperref[op22]{Calcolo fatturato totale} & 1 & Admin \\
    \hline
    21 & \hyperref[op23]{Analisi recensioni e valutazioni} & 1 & Admin \\
    \hline
  \end{tabularx}
  \caption{Numero stimato di operazioni per settimana, con tipo di utente che le effettua}
  \label{tab:operazioni-settimanali}
\end{table}



\section{Analisi delle operazioni}
Di seguito viene riportata un'analisi per alcune delle operazioni principali sul database dell'agriturismo. \\
Per stimare il carico delle operazioni sul database si è deciso di introdurre i seguenti parametri:
\begin{itemize}
	\item $A_{lett}$ = numero di accessi in lettura effettuati durante l'operazione. (\textit{read})
	\item $A_{scr}$ = numero di accessi in scrittura effettuati durante l'operazione. (\textit{write})
	\item $Op_{set}$ = numero medio di volte in cui l'operazione viene eseguita in una settimana (dalla Tabella \ref{tab:operazioni-settimanali}).
\end{itemize}
Nel calcolo degli accessi si stima come doppio il peso degli accessi in scrittura, rispetto a quelli in lettura
\begin{enumerate}
	\item {\large \textbf{Registrazione nuovo utente}} \label{op1}

	      In questa operazione viene gestita la creazione di un nuovo utente del sistema.

	      \begin{table}[H]
		      \centering
		      \small
		      \renewcommand{\arraystretch}{1.15}
		      \begin{tabularx}{0.8\textwidth}{|X|c|c|c|}
			      \hline
			      \rowcolor{gray!20}
			      \textbf{Nome} & \textbf{Tipo} & \textbf{Numero accessi} & \textbf{S/L} \\
			      \hline
			      USER & E & 1 & L \\
			      \hline
			      PERSON & E & 1 & S \\
			      \hline
			      USER & E & 1 & S \\
			      \hline
		      \end{tabularx}
	      \end{table}

	      Il flusso dunque si articola in tre parti principali:
	      \begin{enumerate}
		      \item Verificare che l'\texttt{username} ed l'\texttt{email} non siano già presenti in \texttt{USER}, al fine di evitare quindi duplicati.
		      \item Inserimento dei dati anagrafici della nuova persona in \texttt{PERSON}.
		      \item Creazione dell'utente vero e proprio in \texttt{USER}, collegato alla relativa persona.
	      \end{enumerate}

	      Sono dunque presenti $A_{lettura}=1$ e $A_{scrittura}=2$.

	      Pertanto il \textbf{costo settimanale} è dato da:
	      \begin{align*}
		      C_{tot} & = O_{settimana} \cdot (A_{lett} + 2 \cdot A_{scr}) \\
		              & = 50 \cdot (1 + 2 \cdot 2)                         \\
		              & = 50 \cdot 5 = \mathbf{250}
	      \end{align*}


	\item {\large \textbf{Autenticazione / login}} \label{op2}

	      $$
		      {Op}_{sett} = 350
	      $$

	      \begin{table}[H]
		      \centering
		      \small
		      \renewcommand{\arraystretch}{1.15}
		      \begin{tabularx}{0.8\textwidth}{|X|c|c|c|}
			      \hline
			      \rowcolor{gray!20}
			      \textbf{Nome} & \textbf{Tipo} & \textbf{Numero accessi} & \textbf{S/L} \\
			      \hline
			      USER & E & 1 & L \\
			      \hline
		      \end{tabularx}
	      \end{table}

	      Quindi si ha una sola operazione di lettura.
	      Pertanto $C_{tot} = 350 \cdot 1 = \mathbf{350}$

	\item {\large \textbf{Prenotazione servizio}} \label{op3}
	      $$
		      {Op}_{sett}=125
	      $$
	      In media ogni prenotazione è associata a $2$ servizi,
	      pertanto per ogni operazione di prenotazione si accede a $2$ record in \texttt{SERVICE}
	      e si scrivono $2$ record in \texttt{RESERVATION\_DETAIL}.

	      \begin{table}[H]
		      \centering
		      \small
		      \renewcommand{\arraystretch}{1.15}
		      \begin{tabularx}{0.8\textwidth}{|X|c|c|c|}
			      \hline
			      \rowcolor{gray!20}
			      \textbf{Nome} & \textbf{Tipo} & \textbf{Numero accessi} & \textbf{S/L} \\
			      \hline
			      USER & E & 1 & L \\
			      RESERVATION & E & 1 & S \\
			      SERVICE & E & 2 & L \\
			      RESERVATION\_DETAIL & A & 2 & S \\
			      \hline
		      \end{tabularx}
	      \end{table}

	      Quindi in totale si hanno $A_{lett}=3$ e $A_{scr}=3$. \\
	      Pertanto il costo settimanale è:
	      $$\mathbf{C_{tot}} = 125 \cdot (3+6)=\mathbf{1125}$$



	\item {\large \textbf{Modifica / cancellazione prenotazione}} \label{op4}

	      $$
		      {Op}_{sett} = 30
	      $$
	      In media si hanno $\frac{12000}{8000}=1.5$ \textbf{servizi per prenotazione}.

	      \begin{table}[H]
		      \centering
		      \small
		      \renewcommand{\arraystretch}{1.15}
		      \begin{tabularx}{0.8\textwidth}{|X|c|c|c|}
			      \hline
			      \rowcolor{gray!20}
			      \textbf{Nome} & \textbf{Tipo} & \textbf{Numero accessi} & \textbf{S/L} \\
			      \hline
			      RESERVATION & E & 1 & L \\
			      RESERVATION\_DETAIL & A & 2 & L \\
			      RESERVATION & E & 1 & S \\
			      RESERVATION\_DETAIL & A & 2 & S \\
			      \hline
		      \end{tabularx}
	      \end{table}

	      Quindi in totale si hanno $A_{lett}=3$ e $A_{scr}=3$, questo perché:
	      \begin{itemize}
		      \item Si accede alla tabella \texttt{RESERVATION} per trovare la prenotazione da modificare.
		      \item Si leggono i \texttt{RESERVATION\_DETAIL} collegati (che sono in media 2 per prenotazione).
		      \item Infine si aggiorna o si elimina la prenotazione (se viene cancellata, si devono rimuovere anche i relativi dettagli).
	      \end{itemize}

	      Pertanto, il costo settimanale è:
	      $$
		      \mathbf{C_{tot}} = 30 \cdot (3 + 2 \cdot 3) = \mathbf{270}
	      $$


	\item {\large \textbf{Gestione/aggiunta prodotti}} \label{op5}

	      In questa operazione, lo staff aggiunge nuovi prodotti o modifica quelli esistenti.

	      $$Op_{sett} = 3$$

	      \begin{table}[H]
		      \centering
		      \small
		      \renewcommand{\arraystretch}{1.15}
		      \begin{tabularx}{0.8\textwidth}{|X|c|c|c|}
			      \hline
			      \rowcolor{gray!20}
			      \textbf{Nome} & \textbf{Tipo} & \textbf{Numero accessi} & \textbf{S/L} \\
			      \hline
			      PRODUCT & E & 1 & S \\
			      \hline
		      \end{tabularx}
	      \end{table}

	      Si ha solamente un'operazione di scrittura:
	      $$C_{tot} = 3 \cdot (0 + 2 \cdot 1) = \mathbf{6}$$

	\item {\large \textbf{Visualizzazione storico ordini}} \label{op6}

	      Il cliente visualizza lo storico dei propri ordini.

	      $$Op_{sett} = 15$$

	      \begin{table}[H]
		      \centering
		      \small
		      \renewcommand{\arraystretch}{1.15}
		      \begin{tabularx}{0.8\textwidth}{|X|c|c|c|}
			      \hline
			      \rowcolor{gray!20}
			      \textbf{Nome} & \textbf{Tipo} & \textbf{Numero accessi} & \textbf{S/L} \\
			      \hline
			      ORDERS & E & 1 & L \\
			      ORDER\_DETAIL & A & 3 & L \\
			      PRODUCT & E & 3 & L \\
			      \hline
		      \end{tabularx}
	      \end{table}

	      In media ogni ordine contiene 3 prodotti.
	      Quindi: $A_{lett}=7$, $A_{scr}=0$.
	      $$\mathbf{C_{tot}} = 15 \cdot 7 = \mathbf{105}$$

	\item {\large \textbf{Inserimento recensione}} \label{op7}

	      $$
		      {Op}_{sett} = 40
	      $$

	      \begin{table}[H]
		      \centering
		      \small
		      \renewcommand{\arraystretch}{1.15}
		      \begin{tabularx}{0.8\textwidth}{|X|c|c|c|}
			      \hline
			      \rowcolor{gray!20}
			      \textbf{Nome} & \textbf{Tipo} & \textbf{Numero accessi} & \textbf{S/L} \\
			      \hline
			      USER & E & 1 & L \\
			      SERVICE & E & 1 & L \\
			      REVIEW & E & 1 & S \\
			      \hline
		      \end{tabularx}
	      \end{table}

	      Quindi in totale si hanno $A_{lett}=2$ e $A_{scr}=1$.

	      Questo perché:
	      \begin{itemize}
		      \item si legge l'\texttt{USER} che lascia la recensione;
		      \item si verifica il \texttt{SERVICE} a cui si riferisce;
		      \item si inserisce un record in \texttt{REVIEW}.
	      \end{itemize}

	      Pertanto il costo settimanale è:
	      $$\mathbf{C_{tot}} = 40 \cdot (2 + 2 \cdot 1) = \mathbf{160}$$


	\item {\large \textbf{Gestione inventario / prodotti}} \label{op8}

	      Per gestione inventario si intende l'aggiornamento dei prodotti del ristorante/bar da parte dello staff.
	      $$
		      {Op}_{sett} = 10
	      $$

	      \begin{table}[H]
		      \centering
		      \small
		      \renewcommand{\arraystretch}{1.15}
		      \begin{tabularx}{0.8\textwidth}{|X|c|c|c|}
			      \hline
			      \rowcolor{gray!20}
			      \textbf{Nome} & \textbf{Tipo} & \textbf{Numero accessi} & \textbf{S/L} \\
			      \hline
			      PRODUCT & E & 1 & L \\
			      PRODUCT & E & 1 & S \\
			      \hline
		      \end{tabularx}
	      \end{table}

	      Quindi in totale si hanno $A_{lett}=1$ e $A_{scr}=1$.

	      Questo perché:
	      \begin{itemize}
		      \item prima leggiamo il record del \texttt{PRODUCT} per verificare l'esistenza di dati attuali
		      \item per poi aggiornarlo (se esiste) o inserirlo
	      \end{itemize}

	      Pertanto il costo settimanale è:
	      $$\mathbf{C_{tot}} = 10 \cdot (1 + 2 \cdot 1) = \mathbf{30}$$

	\item {\large \textbf{Creazione evento}} \label{op9}

	      $$
		      {Op}_{sett} = 1
	      $$

	      \begin{table}[H]
		      \centering
		      \small
		      \renewcommand{\arraystretch}{1.15}
		      \begin{tabularx}{0.8\textwidth}{|X|c|c|c|}
			      \hline
			      \rowcolor{gray!20}
			      \textbf{Nome} & \textbf{Tipo} & \textbf{Numero accessi} & \textbf{S/L} \\
			      \hline
			      EVENT & E & 1 & S \\
			      \hline
		      \end{tabularx}
	      \end{table}

	      Questo perchè l'operazione comporta semplicemente l'inserimento di un nuovo \texttt{EVENT}.

	      Quindi si ha solamente un'operazione di scrittura:
	      $$
		      \mathbf{C_{tot}} = 1 \cdot (0 + 2 \cdot 1) = \mathbf{2}
	      $$

	\item {\large \textbf{Iscrizione evento}} \label{op10}

	      $$
		      {Op}_{sett} = 15
	      $$

	      In media ad un evento si iscrivono $\frac{2500}{150}=16.7$ \textbf{iscritti}

	      \begin{table}[H]
		      \centering
		      \small
		      \renewcommand{\arraystretch}{1.15}
		      \begin{tabularx}{0.9\textwidth}{|X|c|c|c|}
			      \hline
			      \rowcolor{gray!20}
			      \textbf{Nome} & \textbf{Tipo} & \textbf{Numero accessi} & \textbf{S/L} \\
			      \hline
			      USER & E & 1 & L \\
			      EVENT & E & 1 & L \\
			      EVENT\_SUBSCRIPTION & A & 1 & S \\
			      \hline
		      \end{tabularx}
	      \end{table}

	      Questo perché:
	      \begin{itemize}
		      \item si legge l'\texttt{USER} che si iscrive;
		      \item si legge l'\texttt{EVENT} scelto;
		      \item si inserisce un nuovo record nella \texttt{EVENT\_SUBSCRIPTION}.
	      \end{itemize}

	      Quindi: $A_{lett}=2$, $A_{scr}=1$.
	      Pertanto il costo settimanale è:
	      $$\mathbf{C_{tot}} = 15 \cdot (2 + 2 \cdot 1) = \mathbf{60}$$


	\item {\large \textbf{Creazione Ordine}} \label{op11}

	      $$
		      {Op}_{sett} = 25
	      $$

	      In media ogni ordine contiene $\frac{45000}{15000}=3$ \textbf{prodotti}.

	      \begin{table}[H]
		      \centering
		      \small
		      \renewcommand{\arraystretch}{1.15}
		      \begin{tabularx}{0.9\textwidth}{|X|c|c|c|}
			      \hline
			      \rowcolor{gray!20}
			      \textbf{Nome} & \textbf{Tipo} & \textbf{Numero accessi} & \textbf{S/L} \\
			      \hline
			      USER & E & 1 & L \\
			      ORDERS & E & 1 & S \\
			      PRODUCT & E & 3 & L \\
			      ORDER\_DETAIL & A & 3 & S \\
			      \hline
		      \end{tabularx}
	      \end{table}

	      Questo perché:
	      \begin{itemize}
		      \item si legge l'\texttt{USER} che effettua l'ordine;
		      \item si inserisce un nuovo record in \texttt{ORDERS};
		      \item si leggono in media 3 \texttt{PRODUCT};
		      \item e si scrivono 3 record in \texttt{ORDER\_DETAIL}.
	      \end{itemize}

	      Quindi: $A_{lett}=4$, $A_{scr}=4$.
	      Pertanto il costo settimanale è:
	      $$\mathbf{C_{tot}} = 25 \cdot (4 + 2 \cdot 4) = \mathbf{300}$$



	\item {\large \textbf{Assegnazione / modifica ruolo dipendenti}} \label{op12}
	      $$
		      Op_{sett} = 1
	      $$

	      \begin{table}[H]
		      \centering
		      \small
		      \renewcommand{\arraystretch}{1.15}
		      \begin{tabularx}{0.8\textwidth}{|X|c|c|c|}
			      \hline
			      \rowcolor{gray!20}
			      \textbf{Nome} & \textbf{Tipo} & \textbf{Numero accessi} & \textbf{S/L} \\
			      \hline
			      EMPLOYEE & E & 1 & L \\
			      EMPLOYEE & E & 1 & S \\
			      \hline
		      \end{tabularx}
	      \end{table}

	      In questa operazione si assegna o modifica il ruolo di un dipendente:
	      \begin{itemize}
		      \item Si legge il \texttt{EMPLOYEE}.
		      \item Si aggiorna il campo \texttt{role}.
	      \end{itemize}

	      Totale: $A_{lett}=1$, $A_{scr}=1$.
	      \[
		      C_{tot} = 1 \cdot (1 + 2 \cdot 1) = \mathbf{3}
	      \]

	\item {\large \textbf{Controllo disponibilità servizi}} \label{op15}

	      Un utente o lo staff verificano la disponibilità di un servizio in un dato periodo.
	      Si assume che gli utenti guardino in media 3 servizi.

	      $$
		      Op_{sett} = 550
	      $$

	      \begin{table}[H]
		      \centering
		      \small
		      \renewcommand{\arraystretch}{1.15}
		      \begin{tabularx}{0.8\textwidth}{|X|c|c|c|}
			      \hline
			      \rowcolor{gray!20}
			      \textbf{Nome} & \textbf{Tipo} & \textbf{Numero accessi} & \textbf{S/L} \\
			      \hline
			      SERVICE & E & 3 & L \\
			      RESERVATION\_DETAIL & A & 15 & L \\
			      \hline
		      \end{tabularx}
	      \end{table}

	      Questo perché:
	      \begin{itemize}
		      \item Si leggono 3 \texttt{SERVICE} per ottenere informazioni sui servizi.
		      \item Per ogni servizio, si leggono in media 5 \texttt{RESERVATION\_DETAIL} per verificare le prenotazioni esistenti nell'intervallo richiesto.
		      \item Operazione di sola lettura.
	      \end{itemize}

	      Quindi: $A_{lett}=18$.
	      $$\mathbf{C_{tot}} = 550 \cdot 18 = \mathbf{9900}$$

	\item {\large \textbf{Generazione report finanziari}} \label{op16}

	      Generazione di report finanziari settimanali.
	      $$Op_{sett} = 1$$

	      \begin{table}[H]
		      \centering
		      \small
		      \renewcommand{\arraystretch}{1.15}
		      \begin{tabularx}{0.8\textwidth}{|X|c|c|c|}
			      \hline
			      \rowcolor{gray!20}
			      \textbf{Nome} & \textbf{Tipo} & \textbf{Numero accessi} & \textbf{S/L} \\
			      \hline
			      ORDERS & E & 15000 & L \\
			      ORDER\_DETAIL & A & 45000 & L \\
			      RESERVATION & E & 8000 & L \\
			      SERVICE & E & 100 & L \\
			      \hline
		      \end{tabularx}
	      \end{table}

	      Totale: $A_{lett} = 15000 + 45000 + 8000 + 100 = 68100$
	      $$\mathbf{C_{tot}} = 1 \cdot 68100 = \mathbf{68100}$$

	\item {\large \textbf{Gestione sottoscrizioni eventi}} \label{op17}

	      Gestione delle iscrizioni agli eventi da parte dello staff.
	      $$Op_{sett} = 5$$

	      \begin{table}[H]
		      \centering
		      \small
		      \renewcommand{\arraystretch}{1.15}
		      \begin{tabularx}{0.8\textwidth}{|X|c|c|c|}
			      \hline
			      \rowcolor{gray!20}
			      \textbf{Nome} & \textbf{Tipo} & \textbf{Numero accessi} & \textbf{S/L} \\
			      \hline
			      EVENT\_SUBSCRIPTION & A & 1 & L \\
			      EVENT\_SUBSCRIPTION & A & 1 & S \\
			      \hline
		      \end{tabularx}
	      \end{table}

	      Quindi ci sono $A_{lett}=1$ e $A_{scr}=1$.
	      $$\mathbf{C_{tot}} = 5 \cdot (1 + 2 \cdot 1) = \mathbf{15}$$

	\item {\large \textbf{Pianificazione turni dipendenti}} \label{op18}
	      $$
		      Op_{sett} = 3
	      $$

	      \begin{table}[H]
		      \centering
		      \small
		      \renewcommand{\arraystretch}{1.15}
		      \begin{tabularx}{0.8\textwidth}{|X|c|c|c|}
			      \hline
			      \rowcolor{gray!20}
			      \textbf{Nome} & \textbf{Tipo} & \textbf{Numero accessi} & \textbf{S/L} \\
			      \hline
			      EMPLOYEE & E & 1 & L \\
			      SHIFT & E & 1 & L \\
			      EMPLOYEE\_SHIFT & A & 1 & S \\
			      \hline
		      \end{tabularx}
	      \end{table}

	      Quindi: $A_{lett}=2$, $A_{scr}=1$.
	      $$C_{tot} = 3 \cdot (2 + 2 \cdot 1) = \mathbf{12}$$


	\item {\large \textbf{Calcolo tasso occupazione camere}} \label{op19}
	      $$
		      {Op}_{sett} = 1
	      $$
	      È un'operazione gestionale volta ad individuare i mesi/scenari a bassa occupazione e pianificare promozioni mirate.

	      \begin{table}[H]
		      \centering
		      \small
		      \renewcommand{\arraystretch}{1.15}
		      \begin{tabularx}{0.8\textwidth}{|X|c|c|c|}
			      \hline
			      \rowcolor{gray!20}
			      \textbf{Nome} & \textbf{Tipo} & \textbf{Numero accessi} & \textbf{S/L} \\
			      \hline
			      RESERVATION & E & 8000 & L \\
			      RESERVATION\_DETAIL & A & 12000 & L \\
			      ROOM & E & 30 & L \\
			      SERVICE & E & 100 & L \\
			      \hline
		      \end{tabularx}
	      \end{table}

	      Questo perché:
	      \begin{itemize}
		      \item si leggono tutte le \texttt{RESERVATION} effettuate,
		      \item si consultano i \texttt{RESERVATION\_DETAIL} per sapere quali camere sono state usate,
		      \item si leggono le entità \texttt{ROOM} e \texttt{SERVICE} per filtrare i soli servizi di tipo "camera".
	      \end{itemize}

	      Si hanno quindi: $A_{\text{lett}} = 8000 + 12000 + 30 + 100 = 20130$
	      Pertanto il costo settimanale è dato da:
	      $$
		      \mathbf{C_{tot}} = 1 \cdot 20130 = \mathbf{20130}
	      $$


	\item {\large \textbf{Analisi eventi più popolari}} \label{op20}

	      È un'operazione volta a identificare gli eventi più richiesti per pianificare meglio le risorse.

	      \begin{table}[H]
		      \centering
		      \small
		      \renewcommand{\arraystretch}{1.15}
		      \begin{tabularx}{0.8\textwidth}{|X|c|c|c|}
			      \hline
			      \rowcolor{gray!20}
			      \textbf{Nome} & \textbf{Tipo} & \textbf{Numero accessi} & \textbf{S/L} \\
			      \hline
			      EVENT\_SUBSCRIPTION & A & 2500 & L \\
			      EVENT & E & 150 & L \\
			      USER & E & 5000 & L \\
			      \hline
		      \end{tabularx}
	      \end{table}

	      Questo perché:
	      \begin{itemize}
		      \item le \texttt{EVENT\_SUBSCRIPTION} permettono di risalire a quali eventi sono stati richiesti,
		      \item la tabella \texttt{EVENT} serve a distinguere i diversi eventi,
		      \item da \texttt{USER} si ricavano informazioni sulla tipologia di clientela.
	      \end{itemize}

	      Si ha quindi $A_{\text{lett}} = 2500 + 150 + 5000 = 7650$
	      $$\mathbf{C_{tot}} = 1 \cdot 7650 = \mathbf{7650}$$


	\item {\large \textbf{Individuazione prodotti più venduti}} \label{op21}
	      $$
		      {Op}_{sett} = 1
	      $$
	      Per promuovere i prodotti più popolari e migliorare l'offerta.

	      \begin{table}[H]
		      \centering
		      \small
		      \renewcommand{\arraystretch}{1.15}
		      \begin{tabularx}{0.8\textwidth}{|X|c|c|c|}
			      \hline
			      \rowcolor{gray!20}
			      \textbf{Nome} & \textbf{Tipo} & \textbf{Numero accessi} & \textbf{S/L} \\
			      \hline
			      ORDER\_DETAIL & A & 45000 & L \\
			      PRODUCT & E & 500 & L \\
			      ORDERS & E & 15000 & L \\
			      \hline
		      \end{tabularx}
	      \end{table}

	      Questo perché:
	      \begin{itemize}
		      \item si leggono le relazioni \texttt{ORDER\_DETAIL} per sapere quanti prodotti sono stati venduti,
		      \item da \texttt{PRODUCT} si individuano i prodotti specifici,
		      \item da \texttt{ORDERS} si ottengono informazioni temporali sulle vendite.
	      \end{itemize}

	      Si ha $A_{\text{lett}} = 45000 + 500 + 15000 = 60500$
	      $$\mathbf{C_{tot}} = 1 \cdot 60500 = \mathbf{60500}$$


	\item {\large \textbf{Calcolo fatturato totale}} \label{op22}
	      $$
		      {Op}_{sett} = 1
	      $$
	      Per valutare la redditività complessiva e pianificare il budget.

	      \begin{table}[H]
		      \centering
		      \small
		      \renewcommand{\arraystretch}{1.15}
		      \begin{tabularx}{0.8\textwidth}{|X|c|c|c|}
			      \hline
			      \rowcolor{gray!20}
			      \textbf{Nome} & \textbf{Tipo} & \textbf{Numero accessi} & \textbf{S/L} \\
			      \hline
			      ORDERS & E & 15000 & L \\
			      ORDER\_DETAIL & A & 45000 & L \\
			      RESERVATION & E & 8000 & L \\
			      SERVICE & E & 100 & L \\
			      \hline
		      \end{tabularx}
	      \end{table}

	      Questo perché:
	      \begin{itemize}
		      \item si leggono gli \texttt{ORDERS} registrati,
		      \item da \texttt{ORDER\_DETAIL} si ottiene la quantità/prezzo dei prodotti acquistati,
		      \item da \texttt{RESERVATION} e \texttt{SERVICE} si calcola il fatturato dai servizi.
	      \end{itemize}

	      Si hanno quindi $A_{\text{lett}} = 15000 + 45000 + 8000 + 100 = 68100$
	      $$\mathbf{C_{tot}} = 1 \cdot 68100 = \mathbf{68100}$$


	\item {\large \textbf{Analisi recensioni e valutazioni}} \label{op23}
	      $$
		      {Op}_{sett} = 1
	      $$
	      Questa operazione è necessaria a capire la soddisfazione dei clienti e migliorare i servizi.

	      \begin{table}[H]
		      \centering
		      \small
		      \renewcommand{\arraystretch}{1.15}
		      \begin{tabularx}{0.8\textwidth}{|X|c|c|c|}
			      \hline
			      \rowcolor{gray!20}
			      \textbf{Nome} & \textbf{Tipo} & \textbf{Numero accessi} & \textbf{S/L} \\
			      \hline
			      REVIEW & E & 3000 & L \\
			      USER & E & 5000 & L \\
			      SERVICE & E & 100 & L \\
			      EVENT & E & 150 & L \\
			      \hline
		      \end{tabularx}
	      \end{table}

	      Questo perché:
	      \begin{itemize}
		      \item si leggono le \texttt{REVIEW} per analizzare commenti e valutazioni,
		      \item si consultano gli \texttt{USER} per segmentare la clientela,
		      \item si leggono \texttt{SERVICE} e \texttt{EVENT} per associare le recensioni ai servizi/eventi.
	      \end{itemize}

	      Si ha $A_{\text{lett}} = 3000 + 5000 + 100 + 150 = 8250$
	      $$\mathbf{C_{tot}} = 1 \cdot 8250 = \mathbf{8250}$$
\end{enumerate}

\section{Analisi delle ridondanze}

\section{Riepilogo operazioni}
\begin{table}[H]
    \centering
    \small
    \renewcommand{\arraystretch}{1.2}
    \begin{tabularx}{\textwidth}{|>{\raggedright\arraybackslash}X|c|c|}
        \hline
        \rowcolor{gray!20}
        \textbf{Operazione} & \textbf{Costo totale/7gg} & \textbf{Tipo Utente} \\
        \hline
        Registrazione nuovo utente & 250 & Cliente \\
        \hline
        Autenticazione / login & 350 & Tutti (Clienti, Staff) \\
        \hline
        Prenotazione servizio & 1125 & Cliente / Reception \\
        \hline
        Modifica / cancellazione prenotazione & 270 & Cliente / Reception \\
        \hline
        Gestione / aggiunta prodotti & 6 & Staff / Admin \\
        \hline
        Visualizzazione storico ordini & 105 & Cliente \\
        \hline
        Inserimento recensione & 160 & Cliente \\
        \hline
        Gestione inventario / prodotti & 30 & Staff \\
        \hline
        Creazione evento & 2 & Staff / Admin \\
        \hline
        Iscrizione evento & 60 & Cliente \\
        \hline
        Creazione ordine & 300 & Cliente / Staff \\
        \hline
        Assegnazione / modifica ruolo dipendenti & 3 & Admin \\
        \hline
        Controllo disponibilità servizi & 9900 & Tutti \\
        \hline
        Generazione report finanziari & 68100 & Admin \\
        \hline
        Gestione sottoscrizioni eventi & 15 & Staff / Admin \\
        \hline
        Pianificazione turni dipendenti & 12 & Admin \\
        \hline
        Calcolo tasso occupazione camere & 20130 & Admin \\
        \hline
        Analisi eventi più popolari & 7650 & Admin / Staff \\
        \hline
        Individuazione prodotti più venduti & 60500 & Admin \\
        \hline
        Calcolo fatturato totale & 68100 & Admin \\
        \hline
        Analisi recensioni e valutazioni & 8250 & Admin \\
        \hline
        \rowcolor{gray!20}
        \textbf{TOTALE} & \textbf{245318} & \\
        \hline
    \end{tabularx}
    \caption{Riepilogo delle operazioni con costi settimanali e tipi di utente}
    \label{tab:riepilogo-operazioni}
\end{table}



\section{Raffinamento dello schema}
In questa sezione vengono illustrati i passaggi di raffinamento dello
schema ER necessari per la sua traduzione nel modello relazionale. Il
processo comprende la rimozione di attributi multivalore, la gestione
delle gerarchie tramite generalizzazione/specializzazione, la
reificazione delle associazioni molti-a-molti e la definizione degli
identificatori principali per ciascuna entità. Questi interventi
garantiscono la coerenza, la semplicità e l'efficienza dello schema
relazionale risultante.

\subsection{Rimozione gerarchie}
Nel nostro schema ER sono presenti alcune gerarchie
(generalizzazioni) che richiedono una traduzione appropriata nel
modello relazionale. Di seguito analizziamo ciascun caso specifico e
le relative scelte implementative.

\subsubsection{Gerarchia di Servizio}
La gerarchia tra SERVICE e le sottoclassi ROOM e RESTAURANT è totale
ed esclusiva: ogni servizio è o una camera o un ristorante. SERVICE
contiene gli attributi comuni, mentre ROOM e RESTAURANT sono tabelle
specializzate collegate tramite chiave esterna. Il campo
\textit{type} in SERVICE identifica il tipo di servizio. Il modello è
facilmente estendibile aggiungendo nuove tabelle specializzate per
altri tipi di servizi.

\subsubsection{Gerarchia di Persona}
La gerarchia tra PERSON, USER ed EMPLOYEE è stata gestita con un
approccio misto. La relazione tra PERSON e USER è stata collassata
verso l'alto: ogni USER corrisponde necessariamente a una PERSONA,
consentendo così di mantenere un'anagrafica centralizzata e priva di
duplicati. Per quanto riguarda il passaggio da USER a EMPLOYEE, si è
preferito sostituire la specializzazione con un'associazione: non
tutti gli USER sono EMPLOYEE, ma solo quelli che ricoprono
effettivamente un ruolo nel personale. Questa soluzione garantisce
una chiara distinzione tra identità anagrafica e ruolo operativo,
semplificando la gestione dei dati e delle funzionalità specifiche
all'interno del sistema.

\subsection{Scelta degli identificatori principali}
Per ogni entità sono stati scelti identificatori che garantiscono
univocità e stabilità nel tempo. La selezione degli identificatori è
stata effettuata in modo da facilitare la gestione delle relazioni,
assicurare la coerenza dei dati e supportare eventuali evoluzioni
future dello schema.

\subsubsection{Identificatori naturali}
Gli identificatori naturali vengono utilizzati nelle tabelle in cui
esiste un attributo intrinsecamente univoco e stabile nel tempo. In
particolare, nella tabella \texttt{PERSON} si adotta il codice
fiscale (\texttt{cf}) come chiave primaria naturale, garantendo
l'unicità dell'anagrafica. Per la tabella \texttt{USER}, lo username
rappresenta l'identificatore naturale, assicurando che ogni utente
sia distinto in modo univoco. Analogamente, la tabella
\texttt{EMPLOYEE} eredita lo username come chiave primaria,
mantenendo la coerenza tra le entità correlate.

\subsubsection{Identificatori artificiali}
Gli identificatori artificiali vengono introdotti quando non è
presente un attributo naturale sufficientemente stabile o univoco,
oppure per semplificare la gestione delle relazioni e delle chiavi
esterne. In questi casi si utilizza tipicamente un campo numerico
auto-incrementale (\texttt{id}) come chiave primaria. Questo
approccio è adottato per entità come \texttt{SERVICE}, \texttt{ORDER},
\texttt{RESERVATION}, \texttt{REVIEW}, \texttt{SHIFT}, \texttt{PRODUCT}
ed \texttt{EVENT}, dove non esiste un attributo intrinseco che garantisca
l'unicità e la stabilità nel tempo.

\subsubsection{Identificatori composti}
Per le entità derivate da reificazioni, sono stati adottati
identificatori composti secondo la notazione dello schema logico:

\begin{itemize}
  \item ORDER\_DETAIL(order, product)
  \item RESERVATION\_DETAIL(reservation, service)
  \item EVENT\_SUBSCRIPTION(event, user)
  \item EMPLOYEE\_SHIFT(employee, shift, shift\_date)
\end{itemize}

\subsection{Scelte progettuali significative}

\subsubsection{Flessibilità del catalogo servizi}
La scelta di non collassare la gerarchia di \texttt{SERVICE} è
significativa dal punto di vista progettuale. Questo approccio consente:

\begin{itemize}
  \item \textbf{Espandibilità}: nuovi tipi di servizi possono essere
    aggiunti semplicemente creando nuove tabelle specializzate.
  \item \textbf{Separazione delle responsabilità}: ogni tipologia di
    servizio mantiene i propri attributi specifici.
  \item \textbf{Efficienza delle query}: il campo \texttt{type} in
    \texttt{SERVICE} permette filtri rapidi senza necessità di join aggiuntivi.
  \item \textbf{Integrità referenziale}: le prenotazioni referenziano
    sempre la tabella \texttt{SERVICE}, garantendo coerenza anche in
    presenza di nuovi servizi.
\end{itemize}

\subsubsection{Gestione storica del personale}
La tabella \texttt{EMPLOYEE\_HISTORY} consente di verificare se un
utente è un ex dipendente: se lo username è presente in
\texttt{EMPLOYEE} l'utente è dipendente attivo, se compare anche in
\texttt{EMPLOYEE\_HISTORY} significa che non lavora più nell'azienda.
L'intreccio tra le due tabelle permette di distinguere tra dipendenti
attuali ed ex dipendenti, soddisfacendo il requisito di audit trail.

\subsubsection{Separazione identità e autenticazione}
La separazione tra \texttt{PERSON} e \texttt{USER} garantisce che i
dati anagrafici siano gestiti indipendentemente dalle credenziali di
accesso. Questo approccio migliora la sicurezza, evita duplicazioni e
semplifica la manutenzione delle informazioni personali e di autenticazione.

\section{Schema relazionale finale}
Dopo aver applicato tutti i raffinamenti, lo schema relazionale
finale è rappresentato dalle seguenti tabelle.

\begin{figure}[H]
  \centering
  \includegraphics[width=\textwidth, trim=0 0 0 0]{./schemas/logic.pdf}
  \caption{Schema relazionale finale}
  \label{fig:schema-relazione}
\end{figure}
\newpage

\chapter{Progettazione della Base di Dati}
Una volta creato il nostro database, riportiamo di seguito una parte
del codice relazionale
utilizzato per la sua implementazione.

\section{Check}
Sono stati utilizzati vincoli di tipo \texttt{CHECK} per definire
alcuni domini e assicurare
semplici proprietà degli attributi. Di seguito un esempio di vincolo
\texttt{CHECK} utilizzato
per assicurare che il prezzo di ogni prodotto sia maggiore zero:

% tex-fmt: off
\begin{sqlcode}[caption={},label={lst:check}]
CREATE TABLE PRODUCT (
    id INT AUTO_INCREMENT PRIMARY KEY,
    name VARCHAR(100) NOT NULL,
    description TEXT NOT NULL,
    price DECIMAL(8,2) NOT NULL CHECK (price > 0)
);
\end{sqlcode}
% tex-fmt: on

\section{Viste}
La seguente vista \texttt{active\_employees} restituisce l'elenco dei
dipendenti attivi,
mostrando per ciascuno username, email, nome, cognome e ruolo. Un
dipendente è considerato
attivo se il suo username non compare nella tabella
\texttt{EMPLOYEE\_HISTORY}, che traccia
lo storico delle variazioni di stato.

% tex-fmt: off
\begin{sqlcode}[caption={},label={lst:view}]
CREATE VIEW active_employees AS
SELECT
    e.username,
    u.email,
    p.name,
    p.surname,
    e.role
FROM EMPLOYEE e
JOIN USER u ON e.username = u.username
JOIN PERSON p ON u.cf = p.cf
WHERE e.username NOT IN (
    SELECT username FROM EMPLOYEE_HISTORY
  );
\end{sqlcode}
% tex-fmt: on

\section{Trigger}
Esempio di trigger per vincolare le recensioni: impedisce di
recensire sia evento che servizio
insieme, e consente la recensione solo se l'utente ha partecipato
all'evento (già svolto) o ha
usufruito del servizio.

% tex-fmt: off
\begin{sqlcode}[caption={},label={lst:trigger}]
DROP TRIGGER IF EXISTS trg_review_before_insert;
DELIMITER $$
CREATE TRIGGER trg_review_before_insert
BEFORE INSERT ON REVIEW
FOR EACH ROW
BEGIN
    DECLARE cnt INT DEFAULT 0;

    IF (NEW.event IS NOT NULL AND NEW.service IS NOT NULL) OR (NEW.event IS NULL AND NEW.service IS NULL) THEN
        SIGNAL SQLSTATE '45000'
            SET MESSAGE_TEXT = 'Set either event or service (not both) for the review.';
    END IF;

    IF NEW.event IS NOT NULL THEN
        SELECT COUNT(*)
            INTO cnt
            FROM EVENT e
            JOIN EVENT_SUBSCRIPTION es
                ON es.event = e.id
             AND es.`user` = NEW.`user`
         WHERE e.id = NEW.event
             AND e.event_date < CURDATE();

        IF cnt = 0 THEN
            SIGNAL SQLSTATE '45000'
                SET MESSAGE_TEXT = 'You can review the event only if you were subscribed and the event date is in the past.';
        END IF;
    END IF;

    IF NEW.service IS NOT NULL THEN
        SELECT COUNT(*)
            INTO cnt
            FROM RESERVATION r
            JOIN RESERVATION_DETAIL rd
                ON rd.reservation = r.id
             AND rd.service = NEW.service
         WHERE r.username = NEW.`user`
             AND rd.end_date < NOW();

        IF cnt = 0 THEN
            SIGNAL SQLSTATE '45000'
                SET MESSAGE_TEXT = 'You can review the service only after you have used it (completed reservation).';
        END IF;
    END IF;
END$$
DELIMITER ;
\end{sqlcode}
% tex-fmt: on

\section{Traduzione delle operazioni}
Vengono presentate le query SQL che implementano le
principali operazioni del sistema agriturismo. Le query sono state
progettate per essere efficienti e sfruttare gli indici e i vincoli
definiti nello schema. Di seguito sono riportate le principali operazioni
raggruppate per area funzionale: statistiche, prenotazioni, recensioni,
iscrizioni eventi, autenticazione e gestione ordini.

\subsection{Visualizzazione statistiche dashboard}
Le seguenti query sono utilizzate per popolare la dashboard
amministrativa con le metriche principali del sistema.

\subsubsection{Top servizi per prenotazioni}
Analizza le prenotazioni per identificare i servizi più richiesti,
distinguendo tra ristoranti e camere attraverso un'articolata
procedura di join e raggruppamento.

% tex-fmt: off
\begin{sqlcode}[caption={}]
SELECT CASE
    WHEN s.type = 'RESTAURANT' THEN CONCAT('Restaurant - ', r.code)
    WHEN s.type = 'ROOM' THEN CONCAT('Room - ', ro.code)
    ELSE s.type
  END AS service_name,
  COUNT(rd.service) AS booking_count
FROM SERVICE AS s
LEFT JOIN RESTAURANT AS r ON s.id = r.service
LEFT JOIN ROOM AS ro ON s.id = ro.service
JOIN RESERVATION_DETAIL AS rd ON s.id = rd.service
GROUP BY s.id, s.type, r.code, ro.code
ORDER BY booking_count DESC
LIMIT 5;
\end{sqlcode}
% tex-fmt: on

\subsubsection{Top prodotti per quantità venduta}
Prodotti più venduti in base alla quantità totale ordinata.

% tex-fmt: off
\begin{sqlcode}[caption={}]
SELECT
  p.name AS product_name,
  SUM(od.quantity) AS total_quantity
FROM PRODUCT AS p
JOIN ORDER_DETAIL AS od ON p.id = od.product
GROUP BY p.id, p.name
ORDER BY total_quantity DESC
LIMIT 5;
\end{sqlcode}
% tex-fmt: on

\newpage
\subsubsection{Top eventi per partecipanti}
Determina gli eventi con il maggior numero di partecipanti totali.

% tex-fmt: off
\begin{sqlcode}[caption={}]
SELECT
  e.title AS event_title,
  e.event_date,
  SUM(es.participants) AS total_participants
FROM EVENT AS e
JOIN EVENT_SUBSCRIPTION AS es ON e.id = es.event
GROUP BY e.id, e.title, e.event_date
ORDER BY total_participants DESC
LIMIT 5;
\end{sqlcode}
% tex-fmt: on

\subsubsection{Top prodotti per fatturato}
Calcola i 5 prodotti che generano il maggior fatturato.

% tex-fmt: off
\begin{sqlcode}[caption={}]
SELECT
  p.name AS product_name,
  SUM(od.quantity * od.unit_price) AS total_revenue
FROM PRODUCT AS p
JOIN ORDER_DETAIL AS od ON p.id = od.product
GROUP BY p.id, p.name
ORDER BY total_revenue DESC
LIMIT 5;
\end{sqlcode}
% tex-fmt: on

\subsubsection{Fatturato totale}
Calcolo del ricavo complessivo generato dalle vendite dei prodotti.

% tex-fmt: off
\begin{sqlcode}[caption={}]
SELECT ROUND(SUM(od.quantity * od.unit_price), 2) as overall_total_revenue
FROM ORDER_DETAIL od;
\end{sqlcode}
% tex-fmt: on

\subsubsection{Statistiche generali del sistema}
Query composita che fornisce un riepilogo completo delle metriche di
sistema tramite sottoselezioni multiple, aggregando dati da diverse
tabelle per offrire una visione d'insieme immediata.

% tex-fmt: off
\begin{sqlcode}[caption={}]
SELECT
  (SELECT COUNT(*) FROM USER) AS total_customers,
  (SELECT COUNT(*) FROM EMPLOYEE) AS total_employees,
  (SELECT COUNT(*) FROM ORDERS) AS total_orders,
  (SELECT ROUND(SUM(od.quantity * od.unit_price), 2) FROM ORDER_DETAIL AS od) AS total_revenue,
  (SELECT COUNT(*) FROM RESERVATION) AS total_reservations;
\end{sqlcode}
% tex-fmt: on

\newpage
\subsection{Prenotazione servizi}
Le principali query per la prenotazione di servizi, come camere e
tavoli al ristorante, includono la verifica della disponibilità, la
creazione della prenotazione e la gestione dei dettagli associati,
garantendo il rispetto dei vincoli di capacità e delle regole
temporali definite dal sistema.

\subsubsection{Verifica disponibilità camere}
La disponibilità delle camere viene verificata analizzando le
prenotazioni esistenti e selezionando solo quelle con capacità
sufficiente e libere nel periodo richiesto. Il controllo si basa
sulla non sovrapposizione temporale tra le prenotazioni già
registrate e l'intervallo desiderato, così da garantire che la camera
sia effettivamente disponibile.

% tex-fmt: off
\begin{sqlcode}[caption={}]
SELECT
  ro.code AS room,
  s.price AS price,
  ro.max_capacity
FROM ROOM AS ro
JOIN SERVICE AS s
  ON s.id = ro.service
WHERE
  ro.max_capacity >= @n_people
  AND ro.service NOT IN (
    SELECT rd.service
    FROM RESERVATION_DETAIL AS rd
    WHERE NOT (rd.end_date <= @start_date OR rd.start_date >= @end_date)
  );
\end{sqlcode}
% tex-fmt: on

\subsubsection{Verifica disponibilità tavoli}
Calcola i posti disponibili considerando le prenotazioni esistenti
che si sovrappongono all'intervallo richiesto, utilizzando un left
join condizionato e una clausola HAVING per filtrare i ristoranti con
posti sufficienti.

% tex-fmt: off
\begin{sqlcode}[caption={}]
SELECT
  r.code AS restaurant,
  s.price AS price,
  r.max_capacity,
  (r.max_capacity - IFNULL(SUM(rd.people), 0)) AS available_seats
FROM RESTAURANT AS r
JOIN SERVICE AS s ON s.id = r.service
LEFT JOIN RESERVATION_DETAIL AS rd ON rd.service = r.service
  AND NOT (rd.end_date <= @start_date OR rd.start_date >= @end_date)
GROUP BY r.service, r.code, s.price, r.max_capacity
HAVING available_seats >= @n_people;
\end{sqlcode}
% tex-fmt: on

\newpage
\subsection{Gestione recensioni}
Gli utenti possono recensire solo eventi conclusi a cui hanno
partecipato o servizi già prenotati e utilizzati, garantendo
valutazioni autentiche.

\subsubsection{Inserimento recensione evento}
Questa query consente di inserire una recensione per un evento solo
se l'utente è iscritto, l'evento si è concluso e non esiste già una
recensione per quell'evento da parte dello stesso utente. In questo
modo si garantisce la correttezza e l'integrità dei dati.

% tex-fmt: off
\begin{sqlcode}[caption={}]
INSERT INTO REVIEW (user, event, rating, comment)
SELECT
  'mrossi' AS user,
  e.id AS event,
  5 AS rating,
  'Amazing experience! Will definitely come again.' AS comment
FROM EVENT AS e
INNER JOIN EVENT_SUBSCRIPTION AS es ON e.id = es.event
  AND es.user = 'mrossi'
WHERE
  e.title = 'Farm Open Day'
  AND e.event_date < CURDATE()
    AND NOT EXISTS (
      SELECT 1
      FROM REVIEW AS r
      WHERE r.user = 'mrossi' AND r.event = e.id
  )
LIMIT 1;
\end{sqlcode}
% tex-fmt: on

\subsubsection{Inserimento recensione servizio}
Consente di inserire una recensione su un servizio solo se l'utente
ha completato una prenotazione per quel servizio e non esiste già una
recensione associata, garantendo così la correttezza referenziale ed
evitando duplicati.

% tex-fmt: off
\begin{sqlcode}[caption={}]
INSERT INTO REVIEW (user, service, rating, comment)
SELECT
  'aneri' AS user,
  s.id AS service,
  4 AS rating,
  'Good service and friendly staff.' AS comment
FROM SERVICE AS s
INNER JOIN RESERVATION_DETAIL AS rd ON s.id = rd.service
INNER JOIN RESERVATION AS r ON rd.reservation = r.id
WHERE
  r.username = 'aneri'
  AND rd.end_date < NOW()
  AND s.type = 'RESTAURANT'
  AND NOT EXISTS (
    SELECT 1
    FROM REVIEW AS rev
    WHERE rev.user = 'aneri' AND rev.service = s.id
  )
LIMIT 1;
\end{sqlcode}
% tex-fmt: on

\subsubsection{Aggiornamento recensione}
Modifica il voto e il commento di una recensione esistente.

% tex-fmt: off
\begin{sqlcode}[caption={}]
UPDATE REVIEW
SET
  rating = 4,
  comment = 'Very good event, but could use more activities. Overall enjoyed it!',
  created_at = NOW()
WHERE
  user = 'mrossi'
    AND event = (
    SELECT id
    FROM EVENT
    WHERE title = 'Farm Open Day'
    )
  AND id IS NOT NULL;
\end{sqlcode}
% tex-fmt: on

\subsection{Gestione iscrizioni eventi}
Le operazioni sulle iscrizioni agli eventi includono la registrazione
di nuovi partecipanti, l'aggiornamento del numero di iscritti e la
cancellazione delle iscrizioni. Il sistema garantisce che il numero
totale di partecipanti non superi la capienza dell'evento e consente
agli utenti di modificare o annullare la propria iscrizione fino
all'inizio dell'evento.

\subsubsection{Update e Iscrizione utente a evento}
Permette a un utente di iscriversi a un evento specificando il numero
di partecipanti. Il meccanismo "ON DUPLICATE KEY UPDATE" trasforma
l'insert in un update che modifica solo il numero di partecipanti,
nel caso in cui l'utente abbia già effettuato un'iscrizione al dato evento.

% tex-fmt: off
\begin{sqlcode}[caption={}]
INSERT INTO EVENT_SUBSCRIPTION (event, user, participants)
SELECT
  e.id,
  u.username,
  4
FROM EVENT AS e
CROSS JOIN USER AS u
WHERE e.title = 'Harvest Festival' AND u.username = 'lblu'
ON DUPLICATE KEY UPDATE participants = 4;
\end{sqlcode}
% tex-fmt: on

\newpage
\subsection{Esecuzione prenotazioni confermate}
Le prenotazioni di servizi (camere e ristoranti) vengono gestite
tramite query che verificano la disponibilità, creano la prenotazione
principale e aggiungono i dettagli relativi al servizio scelto. Il
sistema assicura che non vi siano sovrapposizioni e che la capacità
sia rispettata, garantendo integrità e correttezza dei dati.

\subsubsection{Creazione prenotazione principale}
Crea il record principale di prenotazione per un utente, con un
controllo che evita la creazione di prenotazioni duplicate nella
stessa giornata.

% tex-fmt: off
\begin{sqlcode}[caption={}]
INSERT INTO RESERVATION (username, reservation_date)
SELECT
  'gverdi',
  NOW()
WHERE NOT EXISTS (
  SELECT 1
  FROM RESERVATION
  WHERE
    username = 'gverdi'
    AND DATE(reservation_date) = CURDATE()
  );
\end{sqlcode}
% tex-fmt: on

\subsubsection{Prenotazione tavolo ristorante}
Prenotazione di tavoli al ristorante, con join sulla tabella
RESTAURANT per identificare correttamente il servizio.
% tex-fmt: off
\begin{sqlcode}[caption={}]
SET @new_reservation_id = LAST_INSERT_ID();

INSERT INTO RESERVATION_DETAIL (reservation, service, start_date, end_date, people)
SELECT
  @new_reservation_id AS reservation,
  s.id AS service,
  '2024-01-25 19:00:00' AS start_date,
  '2024-01-25 21:00:00' AS end_date,
  2 AS people
FROM SERVICE AS s
INNER JOIN RESTAURANT AS r ON s.id = r.service
WHERE r.code = 'T01'
LIMIT 1;
\end{sqlcode}
% tex-fmt: on

\newpage
\subsubsection{Prenotazione camera con controllo duplicati}
Gestione della prenotazione di una camera, con creazione della
prenotazione principale e dei dettagli. Sono previsti controlli per
evitare duplicati e viene gestito correttamente l'ID generato per la
prenotazione.

% tex-fmt: off
\begin{sqlcode}[caption={}]
INSERT INTO RESERVATION (username, reservation_date)
SELECT 'fbianchi', DATE_ADD(NOW(), INTERVAL 1 HOUR)
WHERE NOT EXISTS (
  SELECT 1 FROM RESERVATION
  WHERE username = 'fbianchi'
  AND DATE(reservation_date) = CURDATE()
);

SET @room_reservation_id = LAST_INSERT_ID();

INSERT INTO RESERVATION_DETAIL (reservation, service, start_date, end_date, people)
SELECT
  @room_reservation_id as reservation,
  s.id as service,
  '2024-01-26 15:00:00' as start_date,
  '2024-01-28 11:00:00' as end_date,
  2 as people
FROM SERVICE s
INNER JOIN ROOM r ON s.id = r.service
WHERE r.code = 'R03'
LIMIT 1;
\end{sqlcode}
% tex-fmt: on

\newpage
\subsection{Eliminazione prenotazioni e iscrizioni}
Le operazioni di eliminazione permettono agli utenti e agli
amministratori di rimuovere prenotazioni di servizi e iscrizioni agli
eventi, garantendo il rispetto dei vincoli temporali e di integrità
referenziale. È possibile cancellare una prenotazione solo se non è
già iniziata, mentre le iscrizioni agli eventi possono essere
annullate fino all'inizio dell'evento. Queste funzionalità assicurano
una gestione sicura e corretta delle risorse e delle partecipazioni.

\subsubsection{Eliminazione prenotazione servizio}
Permette di cancellare una prenotazione di servizio esistente,
rimuovendo prima i dettagli della prenotazione per rispettare i
vincoli di integrità referenziale e successivamente il record
principale della prenotazione.

% tex-fmt: off
\begin{sqlcode}[caption={}]
DELETE FROM RESERVATION_DETAIL
WHERE reservation = @reservation_id;

DELETE FROM RESERVATION
WHERE id = @reservation_id
AND username = @username;
\end{sqlcode}
% tex-fmt: on

\subsubsection{Eliminazione iscrizione evento}
Rimuove l'iscrizione di un utente a un evento specifico, verificando
che l'iscrizione esista e che l'evento non sia già iniziato; in
questo modo è consentita la cancellazione solo per eventi futuri.

% tex-fmt: off
\begin{sqlcode}[caption={}]
DELETE FROM EVENT_SUBSCRIPTION
WHERE
  user = @username
  AND event = @event_id
  AND EXISTS (
    SELECT 1
    FROM EVENT AS e
    WHERE
      e.id = @event_id
      AND e.event_date > CURDATE()
  );
\end{sqlcode}
% tex-fmt: on

\subsubsection{Eliminazione prenotazione con controllo temporale}
L'eliminazione di una prenotazione è consentita solo se non è già
iniziata; viene implementato un controllo temporale che previene la
cancellazione di prenotazioni in corso o concluse.

% tex-fmt: off
\begin{sqlcode}[caption={}]
DELETE FROM RESERVATION
WHERE
  id = @reservation_id
  AND username = @username
  AND NOT EXISTS (
    SELECT 1
    FROM RESERVATION_DETAIL AS rd
    WHERE
      rd.reservation = @reservation_id
      AND rd.start_date <= NOW()
  );
\end{sqlcode}
% tex-fmt: on

\newpage
\subsection{Autenticazione utente}
L'autenticazione utente consente l'accesso sicuro alla piattaforma
tramite verifica di username e password. Per distinguere tra clienti
e dipendenti, viene utilizzata la vista \texttt{active\_employees}:
se il campo \texttt{role} restituito dalla vista è \texttt{NULL},
l'utente è considerato un cliente normale; se invece è valorizzato,
l'utente è un dipendente attivo e il ruolo viene mostrato.

\subsubsection{Verifica credenziali di login}
La query seguente verifica le credenziali di login e determina il
profilo utente, sfruttando la vista \texttt{active\_employees} per
identificare i dipendenti attivi.

% tex-fmt: off
\begin{sqlcode}[caption={}]
SELECT
  u.username,
  u.email,
  p.name,
  p.surname,
  CASE WHEN ae.role IS NOT NULL THEN 'employee' ELSE 'customer' END AS user_type,
  ae.role AS employee_role
FROM USER AS u
JOIN PERSON AS p ON u.cf = p.cf
LEFT JOIN active_employees AS ae ON u.username = ae.username
WHERE u.username = 'mrossi' OR u.email = 'mrossi@farm.com';
\end{sqlcode}
% tex-fmt: on

\newpage
\subsection{Gestione ordini prodotti}
Gli ordini prodotti vengono gestiti tramite una transazione che crea
l'ordine principale e inserisce i dettagli dei prodotti selezionati,
con quantità e prezzo corrente. Il sistema assicura che ogni ordine
sia associato all'utente e che i dati siano registrati in modo consistente.

\subsubsection{Creazione nuovo ordine}
La seguente procedura crea un nuovo ordine, recupera automaticamente
l'ID generato e inserisce i prodotti con i prezzi correnti, gestendo
tutta la logica di ordine in un'unica sequenza di operazioni.

% tex-fmt: off
\begin{sqlcode}[caption={}]
SET @new_order_id = LAST_INSERT_ID();

INSERT INTO ORDER_DETAIL (order, product, quantity, unit_price)
SELECT
  @new_order_id AS order_id,
  p.id AS product_id,
  3 AS quantity,
  p.price AS unit_price
FROM PRODUCT AS p
WHERE p.name = 'Farm Eggs (12 pcs)'
LIMIT 1;

INSERT INTO ORDER_DETAIL (order, product, quantity, unit_price)
SELECT
  @new_order_id AS order_id,
  p.id AS product_id,
  2 AS quantity,
  p.price AS unit_price
FROM PRODUCT AS p
WHERE p.name = 'Fresh Bread'
LIMIT 1;

INSERT INTO ORDER_DETAIL (order, product, quantity, unit_price)
SELECT
  @new_order_id AS order_id,
  p.id AS product_id,
  1 AS quantity,
  p.price AS unit_price
FROM PRODUCT AS p
WHERE p.name = 'Honey Jar (500g)'
LIMIT 1;
\end{sqlcode}
% tex-fmt: on

\chapter{Progettazione dell'applicazione}
L'applicazione è stata sviluppata con il framework \textbf{Django},
che gestisce routing,
database e autenticazione in modo sicuro e scalabile.

\section{Barra di Navigazione}
La \textbf{barra di navigazione} permette un accesso rapido alle
principali sezioni del sito,
come prodotti, eventi, servizi, area personale e funzioni amministrative.

\begin{figure}[H]
  \centering
  \includegraphics[width=\textwidth, trim=0 0 0 0]{./img/navbar.png}
  \caption{Barra di navigazione}
  \label{fig:navbar}
\end{figure}

\newpage
\subsection*{Login}

Il form di login consente agli utenti registrati di accedere
rapidamente alla piattaforma inserendo username e
password. Il sistema verifica le credenziali e, in caso di errore,
mostra un messaggio di avviso.

\begin{figure}[H]
  \centering
  \includegraphics[width=\textwidth, trim=0 0 0 0]{./img/login.png}
  \vspace{-1em}
  \label{fig:login}
\end{figure}

\subsection*{Registrazione}
Anche per registrarsi è disponibile un form semplice e intuitivo, che
permette agli utenti di creare un nuovo
account inserendo i dati richiesti. Dopo la registrazione, l'utente
potrà accedere a tutte le funzionalità della
piattaforma.

\begin{figure}[H]
  \centering
  \includegraphics[width=\textwidth, trim=0 0 0 0]{./img/register.png}
  \vspace{-1em}
  \label{fig:registrazione}
\end{figure}

\newpage
\section{Interfaccia Utente}
Dopo l'accesso, l'utente potrà visualizzare il profilo, con le
prenotazioni e gli ordini, con
la possibilità di recensire o annullare prenotazioni future.

\begin{figure}[H]
  \centering
  \includegraphics[width=\textwidth, trim=0 0 0 0]{./img/users/profile.png}
  \vspace{-1em}
  \label{fig:profile}
\end{figure}

\subsection*{Servizi}
Dopo aver scelto il servizio da prenotare, è sufficiente inserire i
dati necessari; il sistema
mostrerà la disponibilità aggiornata del servizio selezionato.

\begin{figure}[H]
  \centering
  \includegraphics[width=\textwidth, trim=0 0 0 0]{./img/users/services.png}
  \vspace{-1em}
  \label{fig:services}
\end{figure}

\subsection*{Eventi}
Nella sezione eventi, viene mostrato l'elenco degli eventi
disponibili. L'utente può selezionare
l'evento di interesse, specificare il numero di partecipanti e
procedere con la prenotazione.

\begin{figure}[H]
  \centering
  \includegraphics[width=\textwidth, trim=0 0 0 0]{./img/users/events.png}
  \vspace{-1em}
  \label{fig:events}
\end{figure}

\subsection*{Recensioni}
Gli utenti possono visualizzare tutte le recensioni e filtrarle per
evento o servizio.

\begin{figure}[H]
  \centering
  \includegraphics[width=\textwidth, trim=0 0 0 0]{./img/users/reviews.png}
  \vspace{-1em}
  \label{fig:recensione}
\end{figure}

Il form permette agli utenti di lasciare una recensione su eventi o
servizi a cui hanno partecipato,
inserendo commento e voto. La recensione è consentita solo dopo la
partecipazione effettiva.

\begin{figure}[H]
  \centering
  \includegraphics[width=\textwidth, trim=0 0 0 0]{./img/users/give_review.png}
  \vspace{-1em}
  \label{fig:lascia recensione}
\end{figure}

\subsection*{Prodotti}
La sezione prodotti consente agli utenti di consultare il catalogo
aggiungerli al carrello per
l'acquisto. Il sistema mostra in tempo reale il contenuto del
carrello e il totale dell'ordine.
Fatto il checkout sarà visibile il riepilogo nella sezione profilo.

\begin{figure}[H]
  \centering
  \includegraphics[width=\textwidth, trim=0 0 0 0]{./img/users/products.png}
  \vspace{-1em}
  \label{fig:products}
\end{figure}

\subsection*{Carrello}
Il \textbf{carrello} è una funzionalità applicativa che consente agli
utenti di selezionare e
gestire i prodotti da acquistare prima di confermare l'ordine. Il
carrello non è rappresentato
nel database, ma viene gestito lato applicazione: i prodotti
selezionati vengono memorizzati
temporaneamente fino al checkout, momento in cui viene creato
l'ordine definitivo e
registrato nel sistema.

\begin{figure}[H]
  \centering
  \includegraphics[width=\textwidth, trim=0 0 0 0]{./img/users/cart.png}
  \vspace{-1em}
  \label{fig:cart}
\end{figure}

\newpage
\section{Interfaccia Amministratore}
Per la gestione amministrativa, l'applicazione sfrutta la sezione
\textbf{Django Admin},
che consente agli amministratori di accedere rapidamente a tutte le
tabelle del database,
modificare dati, tramite un'interfaccia web sicura e strutturata.

\begin{figure}[H]
  \centering
  \includegraphics[width=\textwidth, trim=0 0 0 0]{./img/admin/djangoAdmin.png}
  \vspace{-1em}
  \label{fig:django-admin}
\end{figure}

Oltre al pannello standard di Django Admin, è stata realizzata una
pagina web dedicata alla
visualizzazione delle statistiche principali del sistema, come
l'andamento delle vendite, la
partecipazione agli eventi e la presenza del personale. Questa pagina
presenta tabelle
riepilogative.

\begin{figure}[H]
  \centering
  \includegraphics[width=\textwidth, trim=0 0 0 0]{./img/admin/statistic.png}
  \vspace{-1em}
  \label{fig:statistiche}
\end{figure}

\appendix
\chapter{Guida Utente}

\section{Clonazione del repository}
Clonare il progetto da GitHub e accedere alla cartella:

\begin{verbatim}
> git clone https://github.com/alessandrorebosio/D25-farmhouse.git
> cd DB25-farmhouse
\end{verbatim}

\section{Installazione delle dipendenze}

Si consiglia di utilizzare un ambiente virtuale Python per isolare le
dipendenze del progetto.

\begin{verbatim}
> python3 -m venv venv
\end{verbatim}

\noindent Attivazione dell'ambiente virtuale
\begin{verbatim}
# Su Linux/macOS:
> source venv/bin/activate
# Su Windows:
> venv\Scripts\activate
\end{verbatim}

\noindent Installazione delle dipendenze dal file requirements.txt
\begin{verbatim}
> pip install -r requirements.txt
\end{verbatim}

\section{Creazione del database}
Per creare il database MySQL a partire dagli script SQL forniti,
assicurarsi di avere MySQL
installato e in esecuzione.

\begin{verbatim}
> mysql -u root -p < app/sql/db.sql
> mysql -u root -p < app/sql/demo.sql
\end{verbatim}

Verrà richiesta la password dell'utente \texttt{root}. Il comando
eseguirà tutte le
istruzioni SQL contenute nel file \texttt{db.sql}, creando tabelle,
vincoli e dati di
esempio necessari per l'applicazione.

\section{Avvio dell'applicazione}

Per avviare l'applicazione Django, assicurarsi che l'ambiente
virtuale sia attivo e che il database
sia stato creato correttamente.

\begin{verbatim}
> python manage.py migrate
> python manage.py runserver
\end{verbatim}

L'applicazione sarà accessibile all'indirizzo
\url{http://localhost:8000/} tramite browser. Effettuare
il login o la registrazione per iniziare a utilizzare il sistema.

\end{document}
