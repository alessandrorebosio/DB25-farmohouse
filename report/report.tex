\documentclass[a4paper,12pt]{report}

% Codifica, lingua, font
\usepackage[utf8]{inputenc}
\usepackage[T1]{fontenc}
\usepackage[italian]{babel}
\usepackage{lmodern}

% Impaginazione
\usepackage{geometry}

% Grafica, colori, tabelle, float
\usepackage{graphicx}
\usepackage{float}
\usepackage[table]{xcolor}
\usepackage{tabularx}

% Matematica
\usepackage{amsmath}

% Verbatim e listing
\usepackage{fancyvrb}
\usepackage{alltt}
\usepackage{listings}

% Liste
% \usepackage{enumitem}

% Link e riferimenti intelligenti (hyperref prima di cleveref)
\usepackage{hyperref}
% \usepackage[italian]{cleveref}

\geometry{margin=1in}

% colori
\definecolor{codegray}{rgb}{0.5,0.5,0.5}
\definecolor{codeBlue}{HTML}{6495ED}
\definecolor{codegreen}{HTML}{12911F}
\definecolor{backcolour}{rgb}{0.95,0.95,0.92}

% stile listings
\lstdefinestyle{sqlstyle}{
    language=SQL,
    backgroundcolor=\color{backcolour},
    commentstyle=\color{codegreen}\itshape,
    keywordstyle=\color{codeBlue}\bfseries,
    numberstyle=\tiny\color{codegray},
    stringstyle=\color{codeBlue},
    basicstyle=\footnotesize\ttfamily,
    breakatwhitespace=false,
    breaklines=true,
    captionpos=b,
    keepspaces=true,
    numbers=left,
    numbersep=10pt,
    showspaces=false,
    showstringspaces=false,
    showtabs=false
}

% ambiente dedicato
\lstnewenvironment{sqlcode}[1][]{
    \lstset{style=sqlstyle,#1}
}{}

\hypersetup{
    colorlinks=false,
    pdfborder={1 1 1},
    linkbordercolor={1 0 0},
    urlbordercolor={1 0 0},
    citebordercolor={1 0 0},
    pdftitle={Elaborato Basi di Dati},
    pdfauthor={Maisam Noumi, Alessandro Rebosio, Filippo Ricciotti}
}

\title{
    \vspace*{2cm}
    \LARGE Relazione per il corso di \\[0.5cm]
    \textit{Basi di Dati} \\[2cm]
    \Huge\textbf{Agriturismo} \\[2cm]
}

\author{
    \Large
    Alessandro Rebosio \\
    Filippo Ricciotti
}

\date{
    \vspace{1cm}
    \today \\[0.5cm]
    Anno Accademico 2024-2025
}

\begin{document}

\maketitle

\tableofcontents

\appendix
\chapter{Guida Utente}

\section{Clonazione del repository}
Clonare il progetto da GitHub e accedere alla cartella:

\begin{verbatim}
> git clone https://github.com/alessandrorebosio/D25-farmhouse.git
> cd DB25-farmhouse
\end{verbatim}

\section{Installazione delle dipendenze}

Si consiglia di utilizzare un ambiente virtuale Python per isolare le dipendenze del progetto.

\begin{verbatim}
> python3 -m venv venv
\end{verbatim}

\noindent Attivazione dell'ambiente virtuale
\begin{verbatim}
# Su Linux/macOS:
> source venv/bin/activate
# Su Windows:
> venv\Scripts\activate
\end{verbatim}

\noindent Installazione delle dipendenze dal file requirements.txt
\begin{verbatim}
> pip install -r requirements.txt
\end{verbatim}

\section{Creazione del database}

Per creare il database MySQL a partire dagli script SQL forniti, assicurarsi di avere MySQL
installato e in esecuzione.

\begin{verbatim}
> mysql -u root -p < app/sql/db.sql
> mysql -u root -p < app/sql/demo.sql
\end{verbatim}

Verrà richiesta la password dell'utente \texttt{root}. Il comando eseguirà tutte le
istruzioni SQL contenute nel file \texttt{db.sql}, creando tabelle, vincoli e dati di
esempio necessari per l'applicazione.

\section{Avvio dell'applicazione}

Per avviare l'applicazione Django, assicurarsi che l'ambiente virtuale sia attivo e che il database
sia stato creato correttamente.

\begin{verbatim}
> python manage.py migrate
> python manage.py runserver
\end{verbatim}

L'applicazione sarà accessibile all'indirizzo \url{http://localhost:8000/} tramite browser. Effettuare
il login o la registrazione per iniziare a utilizzare il sistema.

\end{document}
